\section{Related Work}
\label{sec:rel-work}


We investigated the current body of knowledge in traditional routing algorithms and artificial intelligence-based routing 
algorithms, as well as their use cases, in the literature. We dissected the widely adopted methods and investigated how they
can be used to reduce the number of devices involved in message transmission. Furthermore, we analysed the benefits of 
integrating various AI algorithms in a network composed of IoT devices and evaluated different algorithms to determine their 
viability in handling uncertainty and energy consumption.

3.1 Studies investigating traditional IoT routing algorithms.

\cite{plageras-psannis:2017} defines the Internet of Things (IoT) as a paradigm that uses information and communication 
technologies to connect physical and/or virtual objects and incorporate them into networks. The Internet of Things (IoT) 
continues to incorporate a significant number of parts and things that stand out for their complexity and variety 
\cite{amrioui-Sofiane-Hamrioui-Camil-Lloret-Jaime-Lorenz-Pascal:2018}. To exploit the benefits of IoT, it is necessary to manage 
the data generated and exchanged in the network in an efficient manner \cite{Tlili-Sihem-Mnasri-Sami-Val-hierry:2022}. 
Routing consists of applying an algorithm to determine the best route between the source and 
the destination among multiple routes. In IoT networks, routing becomes essential because of its exceptional characteristics: 
intelligence, connectivity, dynamic nature, security, and ability to host large-scale devices. The network sends data to gateways 
as hosts and routers\cite{:2019}. Thus, routing policy specifies how routing devices connect to 
the network and circulate information, controlling the best route between two nodes among different routes. The author/cite 
introduced an approach to mitigate packet loss by combining congestion management strategies with routing procedures to control
 inflow from the network layer. This boasts scalability by using a few nodes in routing and discovery. Similarly, combining 
 proactive and reactive features minimises the delay and overhead caused by these protocols when nodes increase. 
 lili-Sihem-Mnasri-Sami-Val-hierry:2022 Several characteristics affect routing policies. Integration of heterogeneity is 
 preferred since the node’s energy affects its power. For faster processing, fewer memories result in high packet loss, 
 hence incorporation. Hybrid mechanisms are encouraged. Another factor is the security of the data. Authentication is 
 encouraged between two devices. Congestion causes packet loss and delays. Multi-routing can be inspiring to balance the 
 load and stop the exhaustion of specific nodes. 
 An efficient and intelligent routing algorithm, SERA (Smart and Self-Organized Routing Algorithm), was proposed by [2] to 
 improve IoT communications. The proposed algorithm is called SSRA, which can select the best route for the packets. 
 Smart cities are defined by the fact that they do not require fixed installation and configuration, in contrast to other types
  of networks. This makes their installation and deployment easy, and their use is solicited by all areas, such as smart cities.
 
  3.2 The AI techniques to overcome routing challenges in a network composed of IoT devices 
  
Applications of artificial intelligence for routing are another way to address challenges in routing. Artificial intelligence
is defined as the science of getting machines to mimic the problem-solving and decision-making abilities of a human brain. 
According to the author, an algorithm called SERA exploits the experience acquired by the sent packets on the used route to 
evaluate its effectiveness factor compared to the other possible routes. This factor is calculated according to the 
communication parameters and the networks’ new situation. SERA recorded better results than three different routing
algorithms regarding QoS parameters and device lifetime. According to the author's findings, using SERA as a routing 
algorithm significantly improves IoT communications. [3], A new smart ration control algorithm (s-RCA) was proposed to 
create an intelligent path in the network between source and destination, which speeds up the medical surgery packets 
used in emergencies. This algorithm will read high-emergency traffic and open a session between the source and destination.
It is configured to wait for some time; if no more packets are received, it will stop booking the path and return to normal
 mode. Node buffering was used to monitor emergency packets. Moreover, this simulation has demonstrated significant 
improvements in network congestion, throughput, delay time, loss ratio, and network overhead. [2] used a self-organized 
routing algorithm (SSRA) to improve the effectiveness of IoT communications in smart cities. SSRA is characterised by its 
ability to use the acquired experience of sending packets along the previous route to select a better future route. 
Smart cities are defined by the fact that they do not require fixed installation and configuration, in contrast to other 
types of networks. This makes their installation and deployment easy, and their use is solicited by all areas, such as 
smart cities. [4] conducted a survey to examine existing works that use artificial intelligence to overcome IoT routing 
challenges. In the case of congested links, the QI-RM algorithm, which used both machine learning algorithms and a data 
flow classification method, improved the following QoS criteria: delay, bandwidth, packet loss rate, and link load 
balancing. However, these performances need to be demonstrated by actual experiments. In 2022, /cite ́ discussed a routing 
optimization problem modelled as a ”multi-constrained shortest path problem” (CSP) to improve performance and optimise resources.
 A multi-CSP problem is an NP-hard problem. The results of the simulations of an NP-hard issue suggest a solution that improves performance in the following areas: 
transmission delay, throughput, energy consumption, packet loss rate, and consumption of bandwidth. However, 
there is a need for improvement to ensure support for failure management and network scalability. 
Several studies have been carried out in the area of IoT routing. The AI-based routing algorithms, or "intelligent routing"
were preferred as they yielded better results in effectively determining the best route in IoT networks than traditional
IoT algorithms, However, actual experiments in the real world need to test their effectiveness. We clearly observed in 
these studies a lot of emphasis has been placed on improving the following criteria: heterogeneity, energy efficiency
,and service quality. There is still limited work on algorithms targeted at minimising the number of devices 
participating in a message transmission. Because IoT devices are less reliable, as indicated by /cite, traditional routing
algorithms may be insufficient in dealing with uncertainty, such as inaccessible devices due to mobility. 
In this study, we developed a message-oriented middleware that incorporates an AI algorithm for a more robust and 
comprehensive routing mechanism for IoT networks.

According to \cite{plageras-psannis:2017}
